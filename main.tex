\documentclass[12pt]{article}
\usepackage{amsmath}
\usepackage{graphicx}
\usepackage[utf8]{inputenc}
\usepackage{biblatex}
\usepackage[a4paper, total={6in, 9in}]{geometry}
\usepackage[acronym]{glossaries}
\usepackage{setspace}
\usepackage{hyperref}
\usepackage{multicol}
\usepackage{subcaption}

\addbibresource{references.bib}

\setstretch{1.5}

\title{Few shot learning in food images with applications in smart ovens using joint embeddings}
\author{Louis Betsch}
% \date{12 September 2023}

\makenoidxglossaries

\newacronym{ijepa}{I-JEPA}{Image-based Joint-Emebbding Predictive Architecture}
\newacronym{tsne}{t-SNE}{t-Distributed Stochastic Neighbor Embedding}
\newacronym{vit}{ViT}{Vision Transformer}
\newacronym[plural=NNs,firstplural=Neural Networks (NNs)]{nn}{NN}{Neural Network}
\newacronym{svm}{SVM}{Support Vector Machine}
\newacronym{knn}{KNN}{k-nearest neighbor}
\newacronym[plural=CNNs,firstplural=Convolutional Neural Networks (CNNs)]{cnn}{CNN}{Convolutional Neural Network}

\begin{document}

\maketitle

\begin{abstract}
	Several breakthroughs in image processing using large datasets were made over the past years and 
	current approaches in the domain of few shot classification in images may benefit by the recent 
	advances in deep learning. Specifically, this work explores the use of the novel self-supervised \gls{ijepa}
	in few shot classification serving as a proof of concept for applications in smart ovens and as a starting point 
	for further improvements. This work compares the usefulness of the resulting representations of the 
	\gls{ijepa} with an established baseline model using a base learner that is trained during validation 
	on few held-out samples. It is shown that learned representations from the \gls{ijepa} improves 
	average accuracy in cross-validation using logistic regression as a base learner. Further research is needed,
	to conclude the usefulness of the \gls{ijepa} as an embedding model in food images. Ideas and an outlook 
	for further improvements are discussed. The source code can be found at \url{https://github.com/roestii/dlvsp_homework}.
\end{abstract}

\clearpage

\tableofcontents
\clearpage

\printnoidxglossary[type=\acronymtype]
\clearpage

\section{Introduction}
\label{sec:introduction}
In recent years, deep learning has gained a lot of attention in image processing with several breakthroughs
\cite{he_deep_2015, krizhevsky_imagenet_2012, tan_efficientnet_2020}. These architectures rely on large labeled datasets. 
Thus, this success can 
partially be attributed to the availability of large public and proprietary datasets like ImageNet \cite{deng_imagenet_2009}. 
In domains where labeled data is sparse architectures with high complexity such as the ones cited above 
remain hard to train \cite{brigato_close_2021}. Therefore, the domain of few shot image classification - 
usually between 1 and 5 samples per class - poses some special challenges and raises the need for other 
approaches. In particular, this work addresses the problem of classifying food images given few held-out 
samples per class for applications such as in smart ovens. A user of an oven may want to save settings for 
a given food, the oven saves a few pictures of that given food inside the oven, and the next time 
the user puts in the same kind of meal, the oven is able to recognize the meal based on the pictures taken 
prior and suggests the saved settings to the user automatically. While there are several challenges to implement 
this use case end-to-end, this work is limited to the classification of meals given a few held-out samples 
per class (meal category).

In contrast to others, this work explores the uses of a novel self-supervised learning architecture called 
\gls{ijepa} \cite{assran_self-supervised_2023} proposed by \citeauthor{assran_self-supervised_2023} 
in few shot classification. The \gls{ijepa} is leveraged as a foundation model and a subsequent base learner is 
used for the classification task.
Addressing the lack of labels in data, self-supervised learning overcomes the issue of labeling by learning
intrinsic features and abstract representations of images \cite{shwartz-ziv_what_2022}.
This approach makes collecting data for training the foundation model easier by not requiring labeling by humans 
or human-assisted systems. The method is compared to a baseline implementation using a ResNet-18 
as the foundation model similar to the approach suggested by
\citeauthor{tian_rethinking_2020} who used a ResNet-12 backbone \cite{tian_rethinking_2020}.
The code and steps to reproduce the results can be found on \url{https://github.com/roestii/dlvsp_homework.git}.


\section{Related works}
\label{sec:related_works}
\citeauthor{li_deep_2023} suggest three major approaches to few shot classification in images \cite{li_deep_2023}:
\begin{enumerate}
	\item{\textbf{Meta learning methods} train a meta-learner on various classification tasks. The idea is to learn
	how to learn and be able to adapt to new classification tasks quickly \cite{li_deep_2023}.}
	\item{\textbf{Transfer learning methods} rely on the assumption that the source and target domain are 
	somewhat related, and a model can be fine-tuned on the target domain \cite{li_deep_2023}.}
	\item{\textbf{Metric learning methods} leverage a feature embedding model and use a subsequent base learner 
	to perform the classification task based on the embeddings \cite{li_deep_2023}.}
\end{enumerate}

Metric learning methods have the advantage of not having to learn new parameters when new classes are added 
and thus overcome the problem of overfitting \cite{li_deep_2023}. \citeauthor{tian_rethinking_2020} use such a metric learning method by training a classification model on the labeled training classes, 
dropping the classification head, and replacing it by a new head trained at test time on the test classes \cite{tian_rethinking_2020}. In particular, they used logistic regression for the classification head \cite{tian_rethinking_2020}. 
\citeauthor{dhillon_baseline_2020} have a similar approach using embedding regularization and cosine similarity 
for classification on the test classes \cite{dhillon_baseline_2020}.

While \cite{tian_rethinking_2020, dhillon_baseline_2020} use one embedding model to extract meaningful feature maps, 
\cite{jiang_few-shot_2020} introduces multi-view representation learning for food classification specifically.
\citeauthor{jiang_few-shot_2020} utilize multiple convolutional networks that predict several different aspects 
of the given food image, such as food category and ingredients and subsequently fuse the individual feature maps to 
make a final class prediction.

While all the methods above use supervised learning for training the initial foundation model, this work 
explores a self-supervised learning method to learn meaningful representations of food images that can 
be used for training a base learner (e.g. logistic regression classifier) to predict the test classes.

The self-supervised learning method \gls{ijepa} is an architecture that consists of an encoder, a predictor, and a 
target encoder \cite{assran_self-supervised_2023}. Instead of relying on \glspl{cnn}, the \gls{ijepa} uses 
the \gls{vit} \cite{dosovitskiy_image_2021} as its basic building block \cite{assran_self-supervised_2023}. 
The underlying idea is to feed an image through the target encoder,
a masked version of that same image through the encoder, and predict the embedding from the original image based 
on the embedding of the masked image using the predictor \cite{assran_self-supervised_2023}. They show that 
by predicting in representation space rather than in the original image space, \gls{ijepa} is capable of 
producing semantic representations while using less training hours than other methods \cite{assran_self-supervised_2023}.


\section{Method}
\label{sec:method}
\subsection{Dataset}
As a dataset the Food-101 dataset \cite{fleet_food-101_2014} is used. It consists of 101 food classes where 
each class comprises 1000 labeled real-world images. These real-world images were not taken in a standardized 
way and suffer high variance. This will most likely affect the performance of the few show classification 
significantly. When training a model for a specific use case, like the one in smart ovens which has a stable environment, one 
might consider using a dataset that is closer to the images queried later during inference. 
Nonetheless, the dataset suits well as a proof of concept testing the robustness of an architecture as well.
The dataset is split into three parts: $D_{train}$ with 70 classes,
$D_{val}$ with 15 classes and, $D_{test}$ comprising the remaining 16 classes.

\subsection{Problem}
The embedding model $\theta$ is trained on the samples in $D_{train}$ either in a supervised or self-supervised manner
minimizing its corresponding loss function $\mathcal{L}$. $\phi$ being the embedding model's parameters:

\begin{equation}
	\theta^*= \underset{\phi}{argmin}\;{\mathcal{L}\;(D_{train}, \phi)}
\end{equation}

During validation 1, 5, or 10 samples from each $D_{val}$ class are processed by $\theta$ to train the 
base learner $L_{base}$ which is used to predict the classes of the remaining samples in $D_{val}$ 
to measure the average accuracy over a 10-fold cross-validation. 
Thus, the objective of the base learner can be formulated as:

\begin{equation}
	\mathcal{B}^* = \underset{\mathcal{B}}{argmin}\;{\mathcal{L}\;(\mathcal{B}, D_{val}, \theta^*)}
\end{equation}

\subsection{Training and validating the embedding model}
As a baseline a ResNet-18 is trained on $D_{train}$ through classification using 
categorical cross-entropy. For validation the classification 
head is dropped, and a base learner is trained on a few held-out embeddings from $D_{val}$. 
For regularization the $L^2$-norm is used. The accuracy is 
measured on the prediction of the base learner for the remaining embeddings of the samples in $D_{val}$.

In case of the \gls{ijepa} foundation model class labels from $D_{train}$ are dropped as they are not required. 
Encoder, target encoder, and predictor are trained on the samples of $D_{train}$. The encoder part of the resulting 
model is used to produce the embeddings for the samples of $D_{val}$. Due to the fact that the \gls{ijepa} encoder relies 
on a \gls{vit} the image is processed into patches with a predefined patch size when feeding it through the encoder 
\cite{dosovitskiy_image_2021}. Each patch corresponds to a region in an image and is further encoded by a positional 
encoding to preserve the location of each individual patch \cite{dosovitskiy_image_2021}. Each patch of an image 
is processed by the encoder in parallel. Thus, the output size is given by the number of patches and the encoding 
of each patch. Subsequently, the loss to be minimized by the \gls{ijepa} is the patch-level $L^2$ distance of the 
embeddings. The desired embedding is computed by concatenating all patch encodings resulting in a single one-dimensional 
vector. The training of the base learner for validation is the same as for the baseline model. 

To speed up training and inference of the base learner all embeddings of $D_{val}$ and $D_{test}$ were precomputed.
Several base learners are evaluated on top of both embedding models using the accuracy on the $D_{val}$. 
The base learner is trained on 1, 5, and 10 held-out samples and average accuracy over a 10-fold cross-validation is 
used as a metric to measure model performance. The best model configuration is evaluated on $D_{test}$.


\section{Experiments}
\label{sec:experiments}
\subsection{Setup}
The images of the Food-101 dataset are cropped to a size of 224x224.
For the baseline model, a ResNet-18 pretrained on the ImageNet1k dataset with a one layer classification head 
is fine-tuned for 15 epochs on the 
$D_{train}$ dataset comprising 70 classes with 70000 images using the Adam optimizer \cite{kingma_adam_2017} with a
learning rate of 0.001 and cross-entropy as the loss function.

The \gls{ijepa} model uses the standard implementation from \url{https://github.com/facebookresearch/ijepa}. The model 
is initialized with pretrained weights from ImageNet1k as well and fine-tuned on the food-101 $D_{train}$ dataset 
for 10 epochs on an A100 GPU. It uses a \gls{vit} as its building block with a patch size of 14.

\subsection{Base learners}
Three different base learners are used on top of the embedding models: Logistic Regression, \gls{knn}, \gls{svm}.
Each of the individual base learners is evaluated for each embedding model. Prior to the training of the base learner, the embeddings 
from the individual foundation models are regularized using the $L^2$-norm. The $k$ hyperparameter in the \gls{knn}	
base learner is chosen based on the amount of samples per class included in the training phase of validation. 
All base learners use the default implementation and settings from \cite{scikit-learn}.



\section{Results}
\label{sec:results}
\begin{table}[h!]
\centering
\begin{tabular}{ c c c c c }
	\hline
	\textbf{Embedding model} & \textbf{Base learner} & \textbf{1-shot} & \textbf{5-shot} & \textbf{10-shot} \\
	\hline
	ResNet-18 & Logistic regression & 0.241 & 0.422 & 0.508 \\
	ResNet-18 & \gls{knn} & 0.221 & 0.322 & 0.418 \\
	ResNet-18 & \gls{svm} & 0.221 & 0.404 & 0.504 \\
	\hline
	\gls{ijepa} & Logistic regression & \underline{\textbf{0.262}} & \underline{\textbf{0.453}} & \underline{\textbf{0.531}} \\
	\gls{ijepa} & \gls{knn} & 0.238 & 0.328 & 0.412 \\
	\gls{ijepa} & \gls{svm} & 0.238 & 0.425 & 0.51 \\
	\hline
\end{tabular}
\caption{Experiments performed on the validation dataset.}
\label{table:validation}
\end{table}

\begin{table}[h!]
\centering
\begin{tabular}{ c c c c c }
	\hline
	\textbf{Embedding model} & \textbf{Base learner} & \textbf{1-shot} & \textbf{5-shot} & \textbf{10-shot} \\
	\hline
	\gls{ijepa} & Logistic regression & 0.197 & 0.352 & 0.434 \\
	\hline
\end{tabular}
\caption{Performance of the best model configuration on the test dataset.}
\label{table:test}
\end{table}



\section{Discussion}
\label{sec:discussion}
Results shown in \cite{tian_rethinking_2020}, even for the baseline (without distilling a ResNet-18) suggest accuracy 
scores on CIFAR around 71.5\% and 5-shot accuracy of 86\%. These results seem far more promising than the 
ones encountered in this work. Other works that use the Food-101 dataset directly like \cite{jiang_few-shot_2020} achieved
one shot accuracy scores of 55\% with multi-view representation learning. This might suggest that the Food-101 
dataset is relatively hard for few show classification. It can be attributed to the use of real-world examples in the 
dataset which suffer high variance \cite{fleet_food-101_2014}.
Furthermore, it has to be noted that the tests in \cite{jiang_few-shot_2020} were only performed on 
five randomly sampled classes \cite{jiang_few-shot_2020}
instead of the 15 or 16 classes that were used in this work making the comparison of the findings hard.
Further work is needed to fully conclude, whether the effectiveness of the self-supervised embedding model
and base learner approach can compete with others in few shot classification of food images.


\section{Conclusion}
\label{sec:conclusion}
This work is dedicated to exploring the use of self-supervised representation learning with \gls{ijepa} 
for few shot classification with possible applications in smart ovens. Two separate architectures, being 
an \gls{ijepa} and a ResNet-18 embedding model, were compared based on their ability in producing meaningful 
representations (embeddings). This ability was tested by using a subsequent base learner that is trained on 
few held-out samples from unseen classes and tested on the remaining samples from those classes. 
It could be shown that the embeddings resulting from the \gls{ijepa} imposes an improvement over the 
baseline ResNet-18 backbone. Nonetheless, leaving room for improvement with a 1-shot average accuracy of 19.7\% 
on 16 held-out test classes using logistic regression as a base learner. Increasing the number of classes 
in the test dataset lead to decreasing accuracy in prediction. Given the fact that adding classes decreases
performance and the total accuracy of few shot classification with the used approach, the approach might not yet 
be suited for use in commercial applications. Nonetheless, it could be shown that these architectures can 
learn some meaningful representations in food images by separating some food classes very distinctively in the plots shown 
\ref{sec:validation_results}. Thus, this work can serve as a starting point for exploring
the use of \gls{ijepa} in image few shot classification. Results from \cite{tian_rethinking_2020} suggest that 
distilling the ResNet model used in their work improved accuracy significantly. The \gls{ijepa} for instance allows
for such distilling in the predictor as well. Future work could explore distilling an \gls{ijepa} to improve performance
further by reusing the predictor part from the previous training cycle. Furthermore, the dimensionality of 
the default \gls{ijepa} might not be suited for subsequent use in a base learner. In this work, the encodings 
corresponding to each patch were just concatenated. One might come up with other methods to reduce the dimensionality
or comprise the result of the encoder in a more meaningful way. For one could use a base learner head for each individual
patch and use an ensemble of base learners to predict the class.



\clearpage

\printbibliography

\end{document}
